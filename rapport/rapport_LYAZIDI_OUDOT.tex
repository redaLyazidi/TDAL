\documentclass[12pt]{article}
\usepackage[utf8]{inputenc}
\usepackage[T1]{fontenc}
\usepackage[francais]{babel}
\usepackage{amsthm}
\usepackage{lmodern}
\usepackage{amsmath}
\usepackage{amsfonts}
\usepackage{pict2e}
\usepackage{listing}
\usepackage{url}
\usepackage{pdfpages}
\usepackage{hyperref}
\usepackage{listings}
\usepackage[DIV15]{typearea}
\title{Rapport sur l'état des armées}
\author{LYAZIDI Reda  \hspace*{1cm} OUDOT Maxime}
\date{\today}
\begin{document}
\maketitle
\tableofcontents

\newpage
\section*{Introduction}

\section{L'Existant}
\subsection{Description}
\subsection{Modifications apportées}
Dans le code de l'existant la méthode "heal" soignait même si le soldat était mort,
dans le but de tests pour le \textbf{Composite}, il a été décidé d'implémenter \underline{\textbf{deux}} méthodes :
\begin{enumerate}
 \item \textbf{infuse\_life} étant la "heal" de départ donc ressuscitant aussi
 \item \textbf{heal} qui ne ressuscite pas donc ne fonctionne que sur les soldats encore en vie.
\end{enumerate}
\end{document}